\setcounter{section}{-1}
\section{Motivation}
Let's  consider
\[
\inf\left\{\int_\Omega |\nabla u|dx \,:\, u \in W^{1,1}(\Omega), \|u\|_{L^1} = K
> 0 \right\} =: m_K,
\]
where $W^{1,1}(\Omega) := \{u \in L^1(\Omega) : Du \in L^1(\Omega, \R^n)\}$ is
the Sobolev space. Then there exists a sequence $(u_j \in
W^{1,1}(\Omega))_{j\in\N}$ such that $\|u_j\|_{L^1(\Omega)}=K$ and $\|\nabla
u_j\|_{L^1(\Omega)} \to m_K$ for $j \to \infty$. Now, in general this does
\emph{not} imply that there is a subsequence $(u_{j_k})_{k\in\N}$ which will
convergence even only weakly to an $v \in W^{1,1}(\Omega)$ with $\|v\|_{L^1} =
K$ and $\|\nabla v \|_{L^1} = m_k$. 

The reason for this is essentially because $L^1$ is not a (topological) dual of
any space, though it is contain in one. \todo{Lacks details}

Another example is the \emph{\textbf{Isoperimetric problem}}:
\[
\min\left\{\sigma_{n-1}(\partial F) \,:\, \text{$F$ with some regularity}, |F| =
K > 0 \right\} =: \gamma_k,
\]
where $|F| =: \mathcal{L}^n(F)$ denotes the $n$-dimensional Lebesgue measure.

\section{Measures}

Let $X$ be a non-empty set. We denote by $\mathcal{P}(X)$ (or $2^X$) the
\emph{power set}, that is, the collection of all subsets of $X$.

\begin{definition}[(outer) measure]
A mapping $\mu : \mathcal{P}(X) \to [0,+\infty]$ satisfying
\begin{enumerate}[(1)]
\item $\mu(\emptyset) = 0$ 
\item $\mu(A) \leq \sum_{k=1}^\infty \mu(A_k)$ if $A \subset
\bigcup_{k=1}^\infty A_k$ \hfill ($\sigma$-subadditivity)
\end{enumerate}
is called an (outer) measure.
\end{definition}

\begin{remark}
The (outer) measure is not decreasing, that is, for $A\subset B$, where $A,B \in
\mathcal{P}(X)$, we have $\mu(A) \leq \mu(B)$. 
\end{remark}

\begin{definition}[Restriction of a measure] If $Y \subset X$, the
\emph{restriction of $\mu$ to $Y$}, denoted by $\mu \mres Y$,
is defined as $(\mu \mres Y)(A) := \mu(Y\cap A)$.
\end{definition}

\begin{definition}[$\mu$-measurable] We call a subset $A \subset X$
$\mu$-measurable if 
\[
\mu(B) = \mu(B\cap A) + \mu(B \setminus A)
\qquad \text{for all} \quad B \subseteq X.
\]
\end{definition}

\begin{remark}
This definition is meaningful since \emph{Vitali} found that there exists a set
$E \subset \R$ which is \emph{not} $\mathcal{L}^1$-measurable.
\end{remark}

\begin{definition}[$\sigma$-algebra]
A subset $\mathfrak{F} \subset \mathcal{P}(X)$ is called a \emph{$\sigma$-algebra of
sets} if holds
\begin{enumerate}[(1)]
\item $\emptyset,X \in \mathfrak{F}$,
\item for $A \in \mathfrak{F}$ also $X\setminus A \in \mathfrak{F}$,
\item for a family $(A_i \in \mathfrak{F})_{i\in I}$ we have 
have $\bigcup_{i\in I} A_i \in \mathfrak{F}$.
\end{enumerate}
\end{definition}

\begin{theorem}
Let $\mu$ be a (outer) measure on $X$, then the restriction to the
$\sigma$-algebra of $\mu$-measurable sets is $\sigma$-additive, that is, if
$(A_j)_{j\in I}$ is a (at most) countable disjoint $\mu$-measurable family of
subsets of $X$, then 
\[
\mu\left(\bigcup_{j \in I} A_j\right) = \sum_{j\in I} \mu\left(A_j\right).
\]
\end{theorem}

\begin{definition} Here we collect some important definitions 
\begin{enumerate}[(1)]
%%%
\item Let $\mathfrak{C} \subset \mathcal{P}(X)$, we call the smallest
$\sigma$-algebra containing $\mathfrak{C}$, the \emph{$\sigma$-algebra generated by
$\mathfrak{C}$}.\footnote{Here $\mathfrak{C}=\setminus\text{mathfrak}\{C\}$} 
%%%
\item The \emph{Borel-algebra} on $\R^n$, denoted by $\mathcal{B}(\R^n)$, is the
$\sigma$-algebra generated by the family of open sets in $\R^n$ (in the standard
topology). The elements of the Borel-algebra are called \emph{Borel sets}.
%%%
\item A (outer) measure $\mu$ in $\R^n$ is called a \emph{Borel measure} if each
Borel sets is $\mu$-measurable.
\item A (outer) measure $\mu$ in $\R^n$ is called \emph{Borel regular} if for
all subsets $A \subseteq \R^n$ there exists a Borel set $B$ such that $A \subseteq
B$ and $\mu(A) = \mu(B)$.
\item A Borel regular measure $\mu$ which is locally finite (e.g. $\mu(K) <
\infty$ for all compact subsets $K \subset \R^n$), is called a \emph{Radon measure}.
\end{enumerate}
\end{definition}

\begin{theorem}
Let $\mu$ be a Radon measure on $\R^n$. We have
\begin{enumerate}[(1)]
\item  for all $A \subseteq \R^n$ holds $\mu(A) = \inf\left\{ \mu(U) \,:\,
U \supset A,\, U \text{ open}\right\}$ \hfill (outer regularity),
\item for all $\mu$-measurable sets $B$ holds 
$\mu(B) = \sup \left\{ \mu (K): K \subset B,\, K \text{ compact}\right\}$ \hfill
(inner regularity).
\end{enumerate}
\end{theorem}

\begin{theorem}[Carath\'eodory's criteria]
Let $\mu$ be a (outer) measure on $\R^n$. If for all $A,B \subset \R^n$ that
satisfy $\dist(A,B) > 0$ we have $\mu(A \cup B) = \mu(A) + \mu(B)$, then $\mu$
is a Borel measure.
\end{theorem}

\begin{examples}~
\begin{enumerate}[(1)]
\item For $x \in \R^n$ we can define that \emph{dirac measure} by  
\[
\delta_x(A) := 
\begin{cases}
1 & x \in A
\\
0 & x \not\in A.
\end{cases}
\]
This is in fact a Radon measure.
%%%
\item We define the \emph{counting measure} by
\[
\# (E) = 
\begin{cases}
\text{card}(E) & \text{if $E$ is finite}
\\
+\infty & \text{otherwise}.
\end{cases}
\]
This measure is Borel regular, but \emph{not} a Radon measure (since it is
clearly not locally finite).
\item The also have the well-known \emph{Lebesgue measure} defined by
\[
\mathcal{L}^n(A) := \inf \left\{\sum_{i=1}^\infty \mathcal{L}^n(Q_i) \mid A
\subset \bigcup_{i=1}^\infty Q_i,\, Q_i \text{ cubes}\right\},
\]
where $\mathcal{L}^n(Q_i)$ is equal to the side length of the cubes $Q_i$ to the
$n$-th power. In particular, we have
\[
\mathcal{L}^1(A) = \inf \left\{ \sum_{i=1}^\infty  \diam C_j \mid A
\subset \bigcup_{i=1}^\infty C_j, \, C_j \subset \R \right\}
\]
and so we can characterize
\[
\mathcal{L}^n = \underbrace{\mathcal{L}^1\times\mathcal{L}^1 \times \dots \times
\mathcal{L}^1}_{n-\text{times}} = \mathcal{L}^{n-1} \times \mathcal{L}^1.
\]
%%%%%
\item (\textbf{Hausdorff measure}) Consider $A \subseteq \R^n$, $\alpha \geq 0$, $\delta \in (0,+\infty]$, we
define the \emph{Hausdorff $\alpha$-dimensional content of $A$} as
\[
\mathcal{H}^\alpha_\delta(A) := \inf\left\{ \sum_{j\in I} \omega_\alpha
\left(\frac{\diam C_j}{2}\right)^\alpha \mid A \subset \bigcup_{j\in I \subset \N} C_j, \, \diam
C_j \leq \delta, C_j \subseteq \R^n \right\},
\]
where the infimum is taking over all the (at most countable) coverings $(C_j \subset
\R^n)_{j\in I}$ of $A$, and set $$\omega_\alpha :=
\frac{\pi^{\frac{\alpha}{2}}}{\Gamma(\frac{\alpha}{2}+1)}.$$ 
Since
$\mathcal{H}^\alpha_\delta$ is a not-increasing function in $\delta$ the
following limit
\[
\mathcal{H}^\alpha(A) := \lim_{\delta \searrow 0} \mathcal{H}^\alpha_\delta (A) =
\sup_{\delta > 0} \mathcal{H}^\alpha_\delta(A)
\]
always exists in the extended real numbers. This limit is defined to be the
\emph{Hausdorff measure}.
\end{enumerate}
\end{examples}

\begin{theorem}[Hausdorff measure is Borel regular]
$\mathcal{H}^\alpha$ is a Borel regular measure on $\R^n$ for all $\alpha \geq
0$.
\end{theorem}

\begin{theorem}[Basic properties of the Hausdorff measure]~
\begin{enumerate}[(1)]
\item $\mathcal{H}^0 = \#$ 
\item $\mathcal{H}^1 = \mathcal{L}^1$ on $\R$ 
\item $\mathcal{H}^\alpha \equiv 0$ for all $\alpha > n$ in $\R^n$. 
\item $\mathcal{H}^\alpha(\lambda A) = \lambda^\alpha \mathcal{H}^\alpha(A)$ for
all $A \subseteq \R^n$ and $\lambda > 0$
\item $\mathcal{H}^\alpha(L(A)) = \mathcal{H}^\alpha(A)$ for all affine
isometry $L : \R^n \to \R^n$.
\end{enumerate}
\end{theorem}

\begin{proof}~
\begin{enumerate}[(1)]
\item Since $\omega_0 = 1$ we have 
\[
\begin{aligned}
\mathcal{H}^0(A) &= \lim_{\delta \searrow 0} \inf \left\{ \sum_{j\in I}
\left(\frac{\diam(C_j)}{2}\right)^0 \mid A \subset \bigcup_{j\in I \subset \N}
C_j, \,
\diam C_j \leq \delta  \right\} 
\\& =
\lim_{\delta \searrow 0} \inf \left\{ \sum_{j\in I}
1 \mid A \subset \bigcup_{j\in I} C_j,\,
\diam C_j \leq \delta  \right\} 
\\&=
\begin{cases}
\text{card}(A) & \text{if $A$ is finite}
\\
\infty & \text{otherwise}
\end{cases}
\end{aligned}
\]
%%
\item We estimate the Lebesgue measure $\mathcal{L}^1$ from both sides by the
Hausdorff measure: Since $\omega_1 = 2 = |(-1,1)|$ we first get 
\[
\begin{aligned}
\mathcal{L}^1(A) 
&= \inf \left\{ \sum_{j\in I} \diam C_j \mid A \subset \bigcup_{j\in I} C_j \right\}
\\ & \leq 
\inf \left\{ \sum_{j\in I} \diam C_j \mid A \subset \bigcup_{j\in I} C_j, \,
\diam C_j \leq \delta \right\}
= 
\mathcal{H}^1_\delta(A),
\end{aligned}
\]
which is true for all $\delta > 0$ so we obtained $\mathcal{L}^1(A) \leq
\mathcal{H}^1(A)$.
\\
Now, we define a partition of $\R$ by $J_{k,\delta} := [k\delta, (k+1)\delta]$
for $k \in \Z$ and first fixed $\delta >0$. These are intervals of diameter
$\delta$ so for every $j \in I$ we get $\diam(C_j \cap I_{k,\delta}) \leq
\delta$. Also we have $\sum_{k\in\Z} \diam(C_j \cap I_{k,\delta}) \leq \diam C_j$,
since $I_{k,\delta}$ are a partition $\R$ of disjoint intervals in $k$. So we
get
\[
\begin{aligned}
\mathcal{L}^1(A) = \inf \left\{ \sum_{j\in I} \diam C_j \mid A \subset
\bigcup_{j\in I} C_j \right\}
\geq \inf \left\{ \sum_{j\in I} \sum_{k\in\Z} \diam (C_j \cap I_{k,\delta}) \mid A \subset
\bigcup_{j\in I} \bigcup_{k\in\Z} C_j\cap I_{k,\delta} \right\}
\end{aligned}
\]
since $\diam (C_j \cap I_{k,\delta} ) \leq \delta$ and after relabeling the
index sets $I$ and $\Z$ to an index set $I^{(k,\delta)}$ this last expressions reads 
\[
\dots = \inf \left\{ \sum_{j \in I^{(k,\delta)}} C_j^{(k)} \mid A \subset \bigcup_{j \in
I^{(k,\delta)}} C_j^{(k)},\, \diam C_j^{(k)} \leq \delta \right\} \geq \mathcal{H}^1_\delta.
\]
And since this is true for every $\delta >0$ we arrive at the claim.
\item \TODO
\end{enumerate}
\end{proof}
